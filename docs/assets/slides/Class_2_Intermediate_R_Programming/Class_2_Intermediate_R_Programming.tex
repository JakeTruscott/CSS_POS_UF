% Options for packages loaded elsewhere
\PassOptionsToPackage{unicode}{hyperref}
\PassOptionsToPackage{hyphens}{url}
%
\documentclass[
  ignorenonframetext,
]{beamer}
\usepackage{pgfpages}
\setbeamertemplate{caption}[numbered]
\setbeamertemplate{caption label separator}{: }
\setbeamercolor{caption name}{fg=normal text.fg}
\beamertemplatenavigationsymbolsempty
% Prevent slide breaks in the middle of a paragraph
\widowpenalties 1 10000
\raggedbottom
\setbeamertemplate{part page}{
  \centering
  \begin{beamercolorbox}[sep=16pt,center]{part title}
    \usebeamerfont{part title}\insertpart\par
  \end{beamercolorbox}
}
\setbeamertemplate{section page}{
  \centering
  \begin{beamercolorbox}[sep=12pt,center]{part title}
    \usebeamerfont{section title}\insertsection\par
  \end{beamercolorbox}
}
\setbeamertemplate{subsection page}{
  \centering
  \begin{beamercolorbox}[sep=8pt,center]{part title}
    \usebeamerfont{subsection title}\insertsubsection\par
  \end{beamercolorbox}
}
\AtBeginPart{
  \frame{\partpage}
}
\AtBeginSection{
  \ifbibliography
  \else
    \frame{\sectionpage}
  \fi
}
\AtBeginSubsection{
  \frame{\subsectionpage}
}
\usepackage{amsmath,amssymb}
\usepackage{iftex}
\ifPDFTeX
  \usepackage[T1]{fontenc}
  \usepackage[utf8]{inputenc}
  \usepackage{textcomp} % provide euro and other symbols
\else % if luatex or xetex
  \usepackage{unicode-math} % this also loads fontspec
  \defaultfontfeatures{Scale=MatchLowercase}
  \defaultfontfeatures[\rmfamily]{Ligatures=TeX,Scale=1}
\fi
\usepackage{lmodern}
\ifPDFTeX\else
  % xetex/luatex font selection
\fi
% Use upquote if available, for straight quotes in verbatim environments
\IfFileExists{upquote.sty}{\usepackage{upquote}}{}
\IfFileExists{microtype.sty}{% use microtype if available
  \usepackage[]{microtype}
  \UseMicrotypeSet[protrusion]{basicmath} % disable protrusion for tt fonts
}{}
\makeatletter
\@ifundefined{KOMAClassName}{% if non-KOMA class
  \IfFileExists{parskip.sty}{%
    \usepackage{parskip}
  }{% else
    \setlength{\parindent}{0pt}
    \setlength{\parskip}{6pt plus 2pt minus 1pt}}
}{% if KOMA class
  \KOMAoptions{parskip=half}}
\makeatother
\usepackage{xcolor}
\newif\ifbibliography
\usepackage{color}
\usepackage{fancyvrb}
\newcommand{\VerbBar}{|}
\newcommand{\VERB}{\Verb[commandchars=\\\{\}]}
\DefineVerbatimEnvironment{Highlighting}{Verbatim}{commandchars=\\\{\}}
% Add ',fontsize=\small' for more characters per line
\usepackage{framed}
\definecolor{shadecolor}{RGB}{248,248,248}
\newenvironment{Shaded}{\begin{snugshade}}{\end{snugshade}}
\newcommand{\AlertTok}[1]{\textcolor[rgb]{0.94,0.16,0.16}{#1}}
\newcommand{\AnnotationTok}[1]{\textcolor[rgb]{0.56,0.35,0.01}{\textbf{\textit{#1}}}}
\newcommand{\AttributeTok}[1]{\textcolor[rgb]{0.13,0.29,0.53}{#1}}
\newcommand{\BaseNTok}[1]{\textcolor[rgb]{0.00,0.00,0.81}{#1}}
\newcommand{\BuiltInTok}[1]{#1}
\newcommand{\CharTok}[1]{\textcolor[rgb]{0.31,0.60,0.02}{#1}}
\newcommand{\CommentTok}[1]{\textcolor[rgb]{0.56,0.35,0.01}{\textit{#1}}}
\newcommand{\CommentVarTok}[1]{\textcolor[rgb]{0.56,0.35,0.01}{\textbf{\textit{#1}}}}
\newcommand{\ConstantTok}[1]{\textcolor[rgb]{0.56,0.35,0.01}{#1}}
\newcommand{\ControlFlowTok}[1]{\textcolor[rgb]{0.13,0.29,0.53}{\textbf{#1}}}
\newcommand{\DataTypeTok}[1]{\textcolor[rgb]{0.13,0.29,0.53}{#1}}
\newcommand{\DecValTok}[1]{\textcolor[rgb]{0.00,0.00,0.81}{#1}}
\newcommand{\DocumentationTok}[1]{\textcolor[rgb]{0.56,0.35,0.01}{\textbf{\textit{#1}}}}
\newcommand{\ErrorTok}[1]{\textcolor[rgb]{0.64,0.00,0.00}{\textbf{#1}}}
\newcommand{\ExtensionTok}[1]{#1}
\newcommand{\FloatTok}[1]{\textcolor[rgb]{0.00,0.00,0.81}{#1}}
\newcommand{\FunctionTok}[1]{\textcolor[rgb]{0.13,0.29,0.53}{\textbf{#1}}}
\newcommand{\ImportTok}[1]{#1}
\newcommand{\InformationTok}[1]{\textcolor[rgb]{0.56,0.35,0.01}{\textbf{\textit{#1}}}}
\newcommand{\KeywordTok}[1]{\textcolor[rgb]{0.13,0.29,0.53}{\textbf{#1}}}
\newcommand{\NormalTok}[1]{#1}
\newcommand{\OperatorTok}[1]{\textcolor[rgb]{0.81,0.36,0.00}{\textbf{#1}}}
\newcommand{\OtherTok}[1]{\textcolor[rgb]{0.56,0.35,0.01}{#1}}
\newcommand{\PreprocessorTok}[1]{\textcolor[rgb]{0.56,0.35,0.01}{\textit{#1}}}
\newcommand{\RegionMarkerTok}[1]{#1}
\newcommand{\SpecialCharTok}[1]{\textcolor[rgb]{0.81,0.36,0.00}{\textbf{#1}}}
\newcommand{\SpecialStringTok}[1]{\textcolor[rgb]{0.31,0.60,0.02}{#1}}
\newcommand{\StringTok}[1]{\textcolor[rgb]{0.31,0.60,0.02}{#1}}
\newcommand{\VariableTok}[1]{\textcolor[rgb]{0.00,0.00,0.00}{#1}}
\newcommand{\VerbatimStringTok}[1]{\textcolor[rgb]{0.31,0.60,0.02}{#1}}
\newcommand{\WarningTok}[1]{\textcolor[rgb]{0.56,0.35,0.01}{\textbf{\textit{#1}}}}
\setlength{\emergencystretch}{3em} % prevent overfull lines
\providecommand{\tightlist}{%
  \setlength{\itemsep}{0pt}\setlength{\parskip}{0pt}}
\setcounter{secnumdepth}{-\maxdimen} % remove section numbering
\PassOptionsToClass{aspectratio=169}{beamer}
\usepackage{../../beamer_style/beamer_style}
\setbeamersize{text margin left=3.5mm,text margin right=3.5mm}
\ifLuaTeX
  \usepackage{selnolig}  % disable illegal ligatures
\fi
\usepackage{bookmark}
\IfFileExists{xurl.sty}{\usepackage{xurl}}{} % add URL line breaks if available
\urlstyle{same}
\hypersetup{
  pdftitle={Intermediate  Programming},
  pdfauthor={Jake S. Truscott, Ph.D},
  hidelinks,
  pdfcreator={LaTeX via pandoc}}

\title{Intermediate \texttt{R} Programming}
\subtitle{POS6933: Computational Social Science}
\author{Jake S. Truscott, Ph.D}
\date{}
\institute{\vspace{-5mm}

University of Florida \newline Spring 2026 \newline \newline \newline
\includegraphics[width=3cm]{../../beamer_style/UF.png} \quad 
\includegraphics[width=3.1cm]{../../images/CSS_POLS_UF_Logo.png}}

\begin{document}
\frame{\titlepage}

\section{Overview}\label{overview}

\begin{frame}{Overview}
\phantomsection\label{overview-1}
\begin{itemize}
\item
  \textbf{Today's Goal}: Improve Effectiveness w/ \texttt{R} Programming

  \par \vspace{2.5mm}
\item
  Random Number Generation in \texttt{R}

  \par  \vspace{2.5mm}
\item
  Loops and Iteration \vspace{2.5mm}

  \par
\item
  Visualizing Data and Relationships Using \texttt{ggplot::()}
\end{itemize}
\end{frame}

\begin{frame}{Getting Started}
\phantomsection\label{getting-started}
\begin{itemize}
\tightlist
\item
  Navigate to main directory folder w/ \texttt{R} environment \& 3
  folders (data, code, practice\_set)

  \par \vspace{2.5mm}
\item
  Open the \texttt{R} environment, then \texttt{File} \(\rightarrow\)
  \text{New File + R Script}

  \par \vspace{2.5mm}
\item
  Run the code emailed earlier today -- this will download code from
  GitHub walkthroughs
\end{itemize}
\end{frame}

\section{Random Number Generation}\label{random-number-generation}

\begin{frame}{Coin Flips}
\phantomsection\label{coin-flips}
\centering

\textbf{What is the probability that any independent coin flip will land on heads?}
\pause

\par \vspace{5mm}

\textbf{Does this change if I flip 50 times?} \pause

\par \vspace{5mm}

\textbf{What about 100 times?}\pause

\par \vspace{5mm}

\textbf{What about 1000 times?}\pause

\par \vspace{5mm}

\textbf{What about 10000 times?}
\end{frame}

\begin{frame}{Coin Flips (Cont.)}
\phantomsection\label{coin-flips-cont.}
\small

\begin{center}\includegraphics{Class_2_Intermediate_R_Programming_files/figure-beamer/coin_10-1} \end{center}
\end{frame}

\begin{frame}{Coin Flips (Cont.)}
\phantomsection\label{coin-flips-cont.-1}
\small

\begin{center}\includegraphics{Class_2_Intermediate_R_Programming_files/figure-beamer/coin_100-1} \end{center}
\end{frame}

\begin{frame}{Coin Flips (Cont.)}
\phantomsection\label{coin-flips-cont.-2}
\small

\begin{center}\includegraphics{Class_2_Intermediate_R_Programming_files/figure-beamer/coin_1k-1} \end{center}
\end{frame}

\begin{frame}{Coin Flips (Cont.)}
\phantomsection\label{coin-flips-cont.-3}
\small

\begin{center}\includegraphics{Class_2_Intermediate_R_Programming_files/figure-beamer/coin_10k-1} \end{center}
\end{frame}

\begin{frame}[fragile]{Coin Flips (Cont.)}
\phantomsection\label{coin-flips-cont.-4}
\begin{itemize}
\tightlist
\item
  We can use \texttt{sample()} to randomly select elements from a vector
\item
  In this case, a coin flip where \(p(heads) = p(tails) = 0.5\) \small
\end{itemize}

\begin{Shaded}
\begin{Highlighting}[]
\NormalTok{sides }\OtherTok{\textless{}{-}} \FunctionTok{c}\NormalTok{(}\StringTok{"Heads"}\NormalTok{, }\StringTok{"Tails"}\NormalTok{)  }\CommentTok{\# Flip Options}
\NormalTok{single\_flip }\OtherTok{\textless{}{-}} \FunctionTok{sample}\NormalTok{(sides, }\AttributeTok{size =} \DecValTok{1}\NormalTok{)  }\CommentTok{\# Single Draw}
\FunctionTok{print}\NormalTok{(single\_flip)}
\end{Highlighting}
\end{Shaded}

\begin{verbatim}
[1] "Tails"
\end{verbatim}
\end{frame}

\begin{frame}[fragile]{6-Sided Die}
\phantomsection\label{sided-die}
\begin{itemize}
\tightlist
\item
  We can use the same approach to ``roll'' a six-sided die. \small
\end{itemize}

\begin{Shaded}
\begin{Highlighting}[]
\NormalTok{sides }\OtherTok{\textless{}{-}} \FunctionTok{c}\NormalTok{(}\DecValTok{1}\SpecialCharTok{:}\DecValTok{6}\NormalTok{)  }\CommentTok{\# 1, 2, 3, 4, 5, 6}
\NormalTok{single\_roll }\OtherTok{\textless{}{-}} \FunctionTok{sample}\NormalTok{(sides, }\AttributeTok{size =} \DecValTok{1}\NormalTok{)  }\CommentTok{\# Single Roll}
\FunctionTok{message}\NormalTok{(}\StringTok{"Result of Single Roll: "}\NormalTok{, single\_roll)}
\end{Highlighting}
\end{Shaded}

\begin{verbatim}
Result of Single Roll: 2
\end{verbatim}

\small
\end{frame}

\begin{frame}[fragile]{Poker Hands}
\phantomsection\label{poker-hands}
\begin{itemize}
\tightlist
\item
  We can even use it to do more complex operations like simulate a
  random draw from 5-card Poker \tiny
\end{itemize}

\begin{Shaded}
\begin{Highlighting}[]
\NormalTok{cards }\OtherTok{\textless{}{-}} \FunctionTok{as.character}\NormalTok{(}\FunctionTok{c}\NormalTok{(}\DecValTok{2}\SpecialCharTok{:}\DecValTok{10}\NormalTok{, }\StringTok{"J"}\NormalTok{, }\StringTok{"Q"}\NormalTok{, }\StringTok{"K"}\NormalTok{, }\StringTok{"A"}\NormalTok{))}
\CommentTok{\# All Card Values}
\NormalTok{suits }\OtherTok{\textless{}{-}} \FunctionTok{c}\NormalTok{(}\StringTok{"Hearts"}\NormalTok{, }\StringTok{"Diamonds"}\NormalTok{, }\StringTok{"Spades"}\NormalTok{, }\StringTok{"Clubs"}\NormalTok{)}
\CommentTok{\# Suits}

\NormalTok{deck }\OtherTok{\textless{}{-}} \FunctionTok{expand.grid}\NormalTok{(}\AttributeTok{value =}\NormalTok{ cards, }\AttributeTok{suit =}\NormalTok{ suits) }\SpecialCharTok{\%\textgreater{}\%}
    \FunctionTok{mutate}\NormalTok{(}\AttributeTok{card =} \FunctionTok{paste}\NormalTok{(value, }\StringTok{"of"}\NormalTok{, suit)) }\SpecialCharTok{\%\textgreater{}\%}
    \FunctionTok{pull}\NormalTok{(card)  }\CommentTok{\# Create a Full Deck}

\NormalTok{random\_draw }\OtherTok{\textless{}{-}} \FunctionTok{sample}\NormalTok{(deck, }\AttributeTok{size =} \DecValTok{5}\NormalTok{, }\AttributeTok{replace =}\NormalTok{ F)}
\CommentTok{\# Random 5{-}Card Draw w/out Replacement}
\end{Highlighting}
\end{Shaded}
\end{frame}

\begin{frame}[fragile]{Poker Hands (Cont.)}
\phantomsection\label{poker-hands-cont.}
\begin{verbatim}
Hand: 
3 of Diamonds
5 of Hearts
10 of Diamonds
2 of Diamonds
Q of Diamonds
\end{verbatim}
\end{frame}

\begin{frame}{Generating Distributions}
\phantomsection\label{generating-distributions}
\begin{itemize}
\item
  What if we wanted to move beyond random selection where each draw or
  iteration exists with equal probability or within a uniform
  distribution?

  \par \vspace{5mm}
\item
  \texttt{R} is very flexible and capable of illustrating sampling
  distributions against expected outcomes
\end{itemize}
\end{frame}

\begin{frame}[fragile]{Generating Distributions (Standard Normal)}
\phantomsection\label{generating-distributions-standard-normal}
\begin{itemize}
\tightlist
\item
  Let's start with 1000 samples from a standard normal distribution
  where \(\mu\) = 50 and \(\sigma\) = 10
\end{itemize}

\begin{Shaded}
\begin{Highlighting}[]
\NormalTok{normal }\OtherTok{\textless{}{-}} \FunctionTok{rnorm}\NormalTok{(}\DecValTok{1000}\NormalTok{, }\AttributeTok{mean =} \DecValTok{50}\NormalTok{, }\AttributeTok{sd =} \DecValTok{10}\NormalTok{)}
\end{Highlighting}
\end{Shaded}

\begin{center}\includegraphics{Class_2_Intermediate_R_Programming_files/figure-beamer/normal_distribution_figure-1} \end{center}
\end{frame}

\begin{frame}{Generating Distributions (Standard Normal)}
\phantomsection\label{generating-distributions-standard-normal-1}
\begin{itemize}
\tightlist
\item
  \textbf{Your Turn}: Generate 1000 draws from a standard normal
  distribution where \(\mu\) = 25 and \(\sigma\) = 10.
\end{itemize}
\end{frame}

\begin{frame}[fragile]{Generating Distributions (Standard Normal -- Ex)}
\phantomsection\label{generating-distributions-standard-normal-ex}
\begin{Shaded}
\begin{Highlighting}[]
\NormalTok{normal }\OtherTok{\textless{}{-}} \FunctionTok{rnorm}\NormalTok{(}\DecValTok{1000}\NormalTok{, }\AttributeTok{mean =} \DecValTok{25}\NormalTok{, }\AttributeTok{sd =} \DecValTok{10}\NormalTok{)}
\end{Highlighting}
\end{Shaded}

\begin{center}\includegraphics{Class_2_Intermediate_R_Programming_files/figure-beamer/normal_distribution_ex_figure-1} \end{center}
\end{frame}

\begin{frame}{Generating Distributions (Exponential -- Ex)}
\phantomsection\label{generating-distributions-exponential-ex}
\begin{itemize}
\tightlist
\item
  \textbf{Your Turn}: Generate 1000 draws from an exponential
  distribution where \texttt{rate} = 2
\end{itemize}
\end{frame}

\begin{frame}[fragile]{Generating Distributions (Exponential -- Ex)}
\phantomsection\label{generating-distributions-exponential-ex-1}
\begin{Shaded}
\begin{Highlighting}[]
\NormalTok{exp }\OtherTok{\textless{}{-}} \FunctionTok{rexp}\NormalTok{(}\DecValTok{1000}\NormalTok{, }\AttributeTok{rate =} \DecValTok{2}\NormalTok{)}
\end{Highlighting}
\end{Shaded}

\begin{center}\includegraphics{Class_2_Intermediate_R_Programming_files/figure-beamer/rexp_ex_figure-1} \end{center}
\end{frame}

\section{Functions}\label{functions}

\begin{frame}{Functions (Basics)}
\phantomsection\label{functions-basics}
\begin{itemize}
\item
  Functions are reusable blocks of code that perform a specific task
  when called, helping avoid repetition.

  \par \vspace{2.5mm}
\item
  They can take arguments (inputs) and return values (outputs), making
  them flexible and generalizable.

  \par \vspace{2.5mm}
\item
  They can also be combined, nested, and used within other functions to
  build complex workflows in a clear, organized way.
\end{itemize}
\end{frame}

\begin{frame}[fragile]{Functions (Basic Syntax)}
\phantomsection\label{functions-basic-syntax}
\small

\begin{Shaded}
\begin{Highlighting}[]
\NormalTok{function\_name }\OtherTok{\textless{}{-}} \ControlFlowTok{function}\NormalTok{(input\_1, input\_2) \{}

    \CommentTok{\# Code to Assume Input\_1 and Input\_2}

    \CommentTok{\# return(Return Output Value or Object)}

\NormalTok{\}}
\end{Highlighting}
\end{Shaded}
\end{frame}

\begin{frame}[fragile]{Functions (Example)}
\phantomsection\label{functions-example}
\small

\begin{Shaded}
\begin{Highlighting}[]
\NormalTok{add\_numbers }\OtherTok{\textless{}{-}} \ControlFlowTok{function}\NormalTok{(x, y) \{}
\NormalTok{    result }\OtherTok{\textless{}{-}}\NormalTok{ x }\SpecialCharTok{+}\NormalTok{ y}
    \FunctionTok{return}\NormalTok{(result)}
\NormalTok{\}}

\FunctionTok{add\_numbers}\NormalTok{(}\DecValTok{5}\NormalTok{, }\DecValTok{3}\NormalTok{)}
\end{Highlighting}
\end{Shaded}
\end{frame}

\begin{frame}{Functions (Basics, Cont.)}
\phantomsection\label{functions-basics-cont.}
\begin{itemize}
\tightlist
\item
  Take some time to try your own!
\item
  Try:

  \begin{itemize}
  \tightlist
  \item
    Random Number Generation
  \item
    Easy Task Completion (e.g., addition, subtraction, etc.)
  \end{itemize}
\end{itemize}
\end{frame}

\section{Loops}\label{loops}

\begin{frame}[fragile]{Loops (Basics)}
\phantomsection\label{loops-basics}
\begin{itemize}
\tightlist
\item
  In \texttt{R}, a for loop is a control structure used to repeat a
  block of code a fixed number of times, iterating over a sequence of
  values. The basic syntax is:
\end{itemize}

\begin{Shaded}
\begin{Highlighting}[]
\ControlFlowTok{for}\NormalTok{ (variable }\ControlFlowTok{in}\NormalTok{ sequence) \{}
    \CommentTok{\# Repeating Code Routine return(Result Value}
    \CommentTok{\# or Object)}
\NormalTok{\}}
\end{Highlighting}
\end{Shaded}
\end{frame}

\begin{frame}[fragile]{Loops (Basics, Cont.)}
\phantomsection\label{loops-basics-cont.}
\begin{itemize}
\tightlist
\item
  For example, we can complete basics rolls of six-sided dice:
\end{itemize}

\footnotesize

\begin{Shaded}
\begin{Highlighting}[]
\NormalTok{rolls }\OtherTok{\textless{}{-}} \FunctionTok{c}\NormalTok{()}

\ControlFlowTok{for}\NormalTok{ (i }\ControlFlowTok{in} \DecValTok{1}\SpecialCharTok{:}\DecValTok{10}\NormalTok{) \{}

\NormalTok{    temp\_roll }\OtherTok{\textless{}{-}} \FunctionTok{sample}\NormalTok{(}\DecValTok{1}\SpecialCharTok{:}\DecValTok{6}\NormalTok{, }\DecValTok{1}\NormalTok{, }\AttributeTok{replace =} \ConstantTok{TRUE}\NormalTok{, }\AttributeTok{prob =} \FunctionTok{rep}\NormalTok{(}\DecValTok{1}\SpecialCharTok{/}\DecValTok{6}\NormalTok{,}
        \DecValTok{6}\NormalTok{))}

\NormalTok{    rolls }\OtherTok{\textless{}{-}} \FunctionTok{c}\NormalTok{(rolls, temp\_roll)}
\NormalTok{\}}

\NormalTok{rolls  }\CommentTok{\# Print}
\end{Highlighting}
\end{Shaded}

\begin{verbatim}
 [1] 4 6 3 1 2 2 3 2 5 5
\end{verbatim}
\end{frame}

\begin{frame}[fragile]{Loops (Basics, Cont.)}
\phantomsection\label{loops-basics-cont.-1}
\begin{itemize}
\tightlist
\item
  We can also conditionally iterate through different values from the
  \texttt{functions} example
\end{itemize}

\footnotesize

\begin{Shaded}
\begin{Highlighting}[]
\NormalTok{add\_numbers }\OtherTok{\textless{}{-}} \ControlFlowTok{function}\NormalTok{(x, y) \{}
\NormalTok{    result }\OtherTok{\textless{}{-}}\NormalTok{ x }\SpecialCharTok{+}\NormalTok{ y}
    \FunctionTok{return}\NormalTok{(result)}
\NormalTok{\}}

\NormalTok{available\_values }\OtherTok{\textless{}{-}} \FunctionTok{c}\NormalTok{(}\DecValTok{1}\SpecialCharTok{:}\DecValTok{10}\NormalTok{)}
\NormalTok{sums }\OtherTok{\textless{}{-}} \FunctionTok{c}\NormalTok{()}

\ControlFlowTok{for}\NormalTok{ (pair }\ControlFlowTok{in} \FunctionTok{seq}\NormalTok{(}\DecValTok{1}\SpecialCharTok{:}\DecValTok{10}\NormalTok{)) \{}
\NormalTok{    temp\_pair }\OtherTok{\textless{}{-}} \FunctionTok{sample}\NormalTok{(available\_values, }\DecValTok{2}\NormalTok{)}
\NormalTok{    sums }\OtherTok{\textless{}{-}} \FunctionTok{c}\NormalTok{(sums, }\FunctionTok{add\_numbers}\NormalTok{(temp\_pair[}\DecValTok{1}\NormalTok{], temp\_pair[}\DecValTok{2}\NormalTok{]))}
\NormalTok{\}}

\NormalTok{sums}
\end{Highlighting}
\end{Shaded}

\begin{verbatim}
 [1]  9  7  9  9 12 15 19 15  8 13
\end{verbatim}
\end{frame}

\begin{frame}[fragile]{Loops (Basics, Cont.)}
\phantomsection\label{loops-basics-cont.-2}
\begin{itemize}
\item
  I can also deal 5 hands from a standard 52-card deck for a game of
  Texas Hold 'Em

  \par \vspace{2.5mm}
\item
  Here's the setup -- What's Next?
\end{itemize}

\footnotesize

\begin{Shaded}
\begin{Highlighting}[]
\FunctionTok{set.seed}\NormalTok{(}\DecValTok{1234}\NormalTok{)  }\CommentTok{\# Seed}
\NormalTok{cards }\OtherTok{\textless{}{-}} \FunctionTok{as.character}\NormalTok{(}\FunctionTok{c}\NormalTok{(}\DecValTok{2}\SpecialCharTok{:}\DecValTok{10}\NormalTok{, }\StringTok{"J"}\NormalTok{, }\StringTok{"Q"}\NormalTok{, }\StringTok{"K"}\NormalTok{, }\StringTok{"A"}\NormalTok{))}
\NormalTok{suits }\OtherTok{\textless{}{-}} \FunctionTok{c}\NormalTok{(}\StringTok{"Hearts"}\NormalTok{, }\StringTok{"Diamonds"}\NormalTok{, }\StringTok{"Spades"}\NormalTok{, }\StringTok{"Clubs"}\NormalTok{)}
\NormalTok{deck }\OtherTok{\textless{}{-}} \FunctionTok{expand.grid}\NormalTok{(}\AttributeTok{value =}\NormalTok{ cards, }\AttributeTok{suit =}\NormalTok{ suits) }\SpecialCharTok{|\textgreater{}}
    \FunctionTok{mutate}\NormalTok{(}\AttributeTok{card =} \FunctionTok{paste}\NormalTok{(value, }\StringTok{"of"}\NormalTok{, suit)) }\SpecialCharTok{|\textgreater{}}
    \FunctionTok{pull}\NormalTok{(card)  }\CommentTok{\# Create a Full Deck}

\NormalTok{hands }\OtherTok{\textless{}{-}} \FunctionTok{lapply}\NormalTok{(}\DecValTok{1}\SpecialCharTok{:}\DecValTok{5}\NormalTok{, }\ControlFlowTok{function}\NormalTok{(x) x)}
\end{Highlighting}
\end{Shaded}
\end{frame}

\begin{frame}[fragile]{Loops (Basics, Cont.)}
\phantomsection\label{loops-basics-cont.-3}
\footnotesize

\begin{Shaded}
\begin{Highlighting}[]
\ControlFlowTok{for}\NormalTok{ (card }\ControlFlowTok{in} \DecValTok{1}\SpecialCharTok{:}\DecValTok{2}\NormalTok{) \{}
    \ControlFlowTok{for}\NormalTok{ (player }\ControlFlowTok{in} \DecValTok{1}\SpecialCharTok{:}\DecValTok{5}\NormalTok{) \{}
\NormalTok{        temp\_player\_card }\OtherTok{\textless{}{-}} \FunctionTok{sample}\NormalTok{(deck, }\DecValTok{1}\NormalTok{, }\AttributeTok{replace =}\NormalTok{ F)}
\NormalTok{        deck }\OtherTok{\textless{}{-}}\NormalTok{ deck[}\SpecialCharTok{!}\NormalTok{deck }\SpecialCharTok{\%in\%}\NormalTok{ temp\_player\_card]}
\NormalTok{        hands[[player]][card] }\OtherTok{\textless{}{-}}\NormalTok{ temp\_player\_card}
\NormalTok{    \}  }\CommentTok{\# For All 5 Players}
\NormalTok{\}  }\CommentTok{\# For Both Cards}

\FunctionTok{do.call}\NormalTok{(cbind, hands)}
\end{Highlighting}
\end{Shaded}

\begin{verbatim}
     [,1]          [,2]            [,3]           
[1,] "3 of Spades" "4 of Diamonds" "J of Diamonds"
[2,] "A of Clubs"  "10 of Hearts"  "6 of Hearts"  
     [,4]         [,5]           
[1,] "2 of Clubs" "10 of Clubs"  
[2,] "6 of Clubs" "7 of Diamonds"
\end{verbatim}
\end{frame}

\begin{frame}{Games of Chance: Blackjack}
\phantomsection\label{games-of-chance-blackjack}
\centering

\textbf{What are the basic rules of Blackjack?}

\par \vspace{5mm}

\includegraphics[width=9cm]{../../images/blackjack.png}
\end{frame}

\begin{frame}{Rules of Blackjack:}
\phantomsection\label{rules-of-blackjack}
\begin{itemize}
\tightlist
\item
  Objective: Beat the dealer by getting closer to 21 without going over
\item
  Card values:

  \begin{itemize}
  \tightlist
  \item
    Number Cards = Face Value
  \item
    Face Cards = 10 (Aces = 1 \emph{or} 11)
  \end{itemize}
\item
  Dealer Rules: Dealer reveals cards after players act and must hit
  until \emph{at least} 17
\item
  Gameplay:

  \begin{itemize}
  \item
    Go Over 21 = \textbf{BUST} (Loss)
  \item
    Tie w/ Dealer = Push (No Win/Loss)
  \item
    Standard Win = \textbf{1:1} (Win Bet x2)
  \item
    Blackjack (Ace + 10-Value Card = \textbf{3:2}
  \end{itemize}
\end{itemize}
\end{frame}

\begin{frame}{Blackjack Exercise}
\phantomsection\label{blackjack-exercise}
\begin{center}
\textbf{Write an \texttt{R} routine to play a round of Blackjack. I will do the same.}
\end{center} \par \vspace{5mm}

\begin{itemize}
\tightlist
\item
  \emph{Hint}: Sample from all 52 cards without replacement\ldots{}
\end{itemize}
\end{frame}

\begin{frame}{Blackjack Exercise (Cont.)}
\phantomsection\label{blackjack-exercise-cont.}
\begin{enumerate}
\item
  What if we play with a four-deck shoe? \pause

  \par \vspace{2.5mm}
\item
  What if I wanted to repeat this process 1,000 times?

  \par \vspace{2.5mm}
\end{enumerate}

\emph{Hint}: Use a loop!
\end{frame}

\begin{frame}{Blackjack Exercise (Cont.)}
\phantomsection\label{blackjack-exercise-cont.-1}
\begin{enumerate}
\item
  Assume I begin with \$1000 every day and bet \$100 each game (though
  I'll only play 10 hands\ldots). Over 100 days, approximately how much
  money am I left with? \emph{Note}: If I run out of money on a given
  day, I'm done -- also, each day restarts with \$1000 but previous
  day's leftover sum is added to aggregate winnings.

  \par \vspace{2.5mm}
\item
  What if I start with \$1000 but don't replace the money every
  day\ldots{} How much will I have after 10 days? 50 days?

  \par \vspace{2.5mm}
\item
  Take some time then play around with blackjack\_simulation.R
\end{enumerate}
\end{frame}

\begin{frame}{Roulette Exercise}
\phantomsection\label{roulette-exercise}
\begin{itemize}
\tightlist
\item
  Head over to Course GitHub (Intermediate Programming in \texttt{R})
\item
  Bottom of \texttt{Number Generation \& Loops}
\end{itemize}
\end{frame}

\section{Data Visualization}\label{data-visualization}

\begin{frame}{Data Visualization}
\begin{itemize}
\item
  \texttt{ggplot()} is an incredibly flexible visualization tool

  \par \vspace{2.5mm}
\item
  There's a balance between professional \& \textit{noisy}
  visualizations

  \par \vspace{2.5mm}
\item
  Some journals \& reviewers are more critical than others

  \par \vspace{2.5mm}
\item
  \textbf{Goal}: \textit{Publication-ready} visualizations capable of
  relaying inferential value on its own
\end{itemize}
\end{frame}

\begin{frame}[fragile]{My Default ggplot() Aesthetics}
\phantomsection\label{my-default-ggplot-aesthetics}
\scriptsize

\begin{Shaded}
\begin{Highlighting}[]
\NormalTok{default\_ggplot\_theme  }\OtherTok{\textless{}{-}} \FunctionTok{theme\_minimal}\NormalTok{(}\AttributeTok{base\_size =} \DecValTok{12}\NormalTok{) }\SpecialCharTok{+}
  \FunctionTok{theme}\NormalTok{(}
    \AttributeTok{plot.title =} \FunctionTok{element\_text}\NormalTok{(}\AttributeTok{hjust =} \FloatTok{0.5}\NormalTok{, }\AttributeTok{size =} \DecValTok{12}\NormalTok{),}
    
    \AttributeTok{axis.title =} \FunctionTok{element\_text}\NormalTok{(}\AttributeTok{size =} \DecValTok{12}\NormalTok{, }\AttributeTok{colour =} \StringTok{\textquotesingle{}black\textquotesingle{}}\NormalTok{),}
    
    \AttributeTok{axis.text =} \FunctionTok{element\_text}\NormalTok{(}\AttributeTok{size =} \DecValTok{10}\NormalTok{, }\AttributeTok{colour =} \StringTok{\textquotesingle{}black\textquotesingle{}}\NormalTok{),}
    
    \AttributeTok{panel.background =} \FunctionTok{element\_rect}\NormalTok{(}\AttributeTok{linewidth =} \DecValTok{1}\NormalTok{, }\AttributeTok{colour =} \StringTok{\textquotesingle{}black\textquotesingle{}}\NormalTok{, }\AttributeTok{fill =} \ConstantTok{NA}\NormalTok{), }
    
    \AttributeTok{legend.position =} \StringTok{\textquotesingle{}bottom\textquotesingle{}}\NormalTok{, }
    
    \AttributeTok{legend.background =} \FunctionTok{element\_rect}\NormalTok{(}\AttributeTok{linewidth =} \DecValTok{1}\NormalTok{, }\AttributeTok{colour =} \StringTok{\textquotesingle{}black\textquotesingle{}}\NormalTok{, }\AttributeTok{fill =} \ConstantTok{NA}\NormalTok{), }
\NormalTok{  )}
\end{Highlighting}
\end{Shaded}
\end{frame}

\begin{frame}[fragile]{My Default ggplot() Aesthetics}
\phantomsection\label{my-default-ggplot-aesthetics-1}
\scriptsize

\begin{Shaded}
\begin{Highlighting}[]
\FunctionTok{set.seed}\NormalTok{(}\DecValTok{1234}\NormalTok{)}

\NormalTok{sample\_data }\OtherTok{\textless{}{-}} \FunctionTok{tibble}\NormalTok{(}\AttributeTok{x =} \FunctionTok{c}\NormalTok{(}\DecValTok{1}\SpecialCharTok{:}\DecValTok{100}\NormalTok{), }\AttributeTok{y =} \FunctionTok{rnorm}\NormalTok{(}\DecValTok{100}\NormalTok{,}
    \AttributeTok{mean =} \FloatTok{0.75}\NormalTok{, }\AttributeTok{sd =} \FloatTok{0.33}\NormalTok{))}

\FunctionTok{summary}\NormalTok{(sample\_data)}
\end{Highlighting}
\end{Shaded}

\begin{verbatim}
       x                y           
 Min.   :  1.00   Min.   :-0.02408  
 1st Qu.: 25.75   1st Qu.: 0.45454  
 Median : 50.50   Median : 0.62307  
 Mean   : 50.50   Mean   : 0.69827  
 3rd Qu.: 75.25   3rd Qu.: 0.90550  
 Max.   :100.00   Max.   : 1.59117  
\end{verbatim}
\end{frame}

\begin{frame}[fragile]{My Default ggplot() Aesthetics}
\phantomsection\label{my-default-ggplot-aesthetics-2}
\scriptsize

\begin{Shaded}
\begin{Highlighting}[]
\NormalTok{sample\_data }\SpecialCharTok{\%\textgreater{}\%}
    \FunctionTok{ggplot}\NormalTok{(}\FunctionTok{aes}\NormalTok{(}\AttributeTok{x =}\NormalTok{ x, }\AttributeTok{y =}\NormalTok{ y)) }\SpecialCharTok{+} \FunctionTok{geom\_point}\NormalTok{() }\SpecialCharTok{+} \FunctionTok{geom\_smooth}\NormalTok{(}\AttributeTok{method =} \StringTok{"lm"}\NormalTok{,}
    \AttributeTok{formula =} \StringTok{"y\textasciitilde{}x"}\NormalTok{)}
\end{Highlighting}
\end{Shaded}

\begin{center}\includegraphics{Class_2_Intermediate_R_Programming_files/figure-beamer/default_ggplot_application_2-1} \end{center}
\end{frame}

\begin{frame}{Adding Default Aes}
\phantomsection\label{adding-default-aes}
\scriptsize

\begin{center}\includegraphics{Class_2_Intermediate_R_Programming_files/figure-beamer/default_ggplot_application_3-1} \end{center}
\end{frame}

\section{Stargazer}\label{stargazer}

\begin{frame}[fragile]{Stargazer}
\phantomsection\label{stargazer-1}
\scriptsize

\begin{Shaded}
\begin{Highlighting}[]
\FunctionTok{library}\NormalTok{(stargazer)}
\NormalTok{temp\_lm }\OtherTok{\textless{}{-}} \FunctionTok{lm}\NormalTok{(Sepal.Length }\SpecialCharTok{\textasciitilde{}}\NormalTok{ Sepal.Width }\SpecialCharTok{+}\NormalTok{ Petal.Length }\SpecialCharTok{+}
\NormalTok{    Petal.Width, }\AttributeTok{data =}\NormalTok{ iris)}

\NormalTok{stargazer}\SpecialCharTok{::}\FunctionTok{stargazer}\NormalTok{(temp\_lm, }\AttributeTok{type =} \StringTok{"text"}\NormalTok{, }\AttributeTok{omit.stat =} \FunctionTok{c}\NormalTok{(}\StringTok{"ser"}\NormalTok{,}
    \StringTok{"f"}\NormalTok{, }\StringTok{"adj.rsq"}\NormalTok{), }\AttributeTok{dep.var.caption =} \StringTok{""}\NormalTok{)}
\end{Highlighting}
\end{Shaded}
\end{frame}

\begin{frame}[fragile]{Stargazer -- Text Example}
\phantomsection\label{stargazer-text-example}
\scriptsize

\begin{verbatim}

========================================
                    Sepal.Length        
----------------------------------------
Sepal.Width           0.651***          
                       (0.067)          
                                        
Petal.Length          0.709***          
                       (0.057)          
                                        
Petal.Width           -0.556***         
                       (0.128)          
                                        
Constant              1.856***          
                       (0.251)          
                                        
----------------------------------------
Observations             150            
R2                      0.859           
========================================
Note:        *p<0.1; **p<0.05; ***p<0.01
\end{verbatim}
\end{frame}

\begin{frame}[fragile]{Stargazer -- Latex Example (type = `latex')}
\phantomsection\label{stargazer-latex-example-type-latex}
\tiny

\begin{verbatim}

% Table created by stargazer v.5.2.3 by Marek Hlavac, Social Policy Institute. E-mail: marek.hlavac at gmail.com
% Date and time: Mon, Jan 26, 2026 - 12:30:33 PM
\begin{table}[!htbp] \centering 
  \caption{} 
  \label{} 
\begin{tabular}{@{\extracolsep{5pt}}lc} 
\\[-1.8ex]\hline 
\hline \\[-1.8ex] 
\\[-1.8ex] & Sepal.Length \\ 
\hline \\[-1.8ex] 
 Sepal.Width & 0.651$^{***}$ \\ 
  & (0.067) \\ 
  & \\ 
 Petal.Length & 0.709$^{***}$ \\ 
  & (0.057) \\ 
  & \\ 
 Petal.Width & $-$0.556$^{***}$ \\ 
  & (0.128) \\ 
  & \\ 
 Constant & 1.856$^{***}$ \\ 
  & (0.251) \\ 
  & \\ 
\hline \\[-1.8ex] 
Observations & 150 \\ 
R$^{2}$ & 0.859 \\ 
\hline 
\hline \\[-1.8ex] 
\textit{Note:}  & \multicolumn{1}{r}{$^{*}$p$<$0.1; $^{**}$p$<$0.05; $^{***}$p$<$0.01} \\ 
\end{tabular} 
\end{table} 
\end{verbatim}
\end{frame}

\begin{frame}{Stargazer -- Latex Example Rendered}
\phantomsection\label{stargazer-latex-example-rendered}
\tiny
\begin{table}[!htbp] \centering 
  \caption{Rendered \texttt{stargazer} Table} 
  \label{} 
\begin{tabular}{@{\extracolsep{5pt}}lc} 
\\[-1.8ex]\hline 
\hline \\[-1.8ex] 
\\[-1.8ex] & Sepal.Length \\ 
\hline \\[-1.8ex] 
 Sepal.Width & 0.651$^{***}$ \\ 
  & (0.067) \\ 
  & \\ 
 Petal.Length & 0.709$^{***}$ \\ 
  & (0.057) \\ 
  & \\ 
 Petal.Width & $-$0.556$^{***}$ \\ 
  & (0.128) \\ 
  & \\ 
 Constant & 1.856$^{***}$ \\ 
  & (0.251) \\ 
  & \\ 
\hline \\[-1.8ex] 
Observations & 150 \\ 
R$^{2}$ & 0.859 \\ 
\hline 
\hline \\[-1.8ex] 
\textit{Note:}  & \multicolumn{1}{r}{$^{*}$p$<$0.1; $^{**}$p$<$0.05; $^{***}$p$<$0.01} \\ 
\end{tabular} 
\end{table}
\end{frame}

\begin{frame}{Stargazer -- Multiple Models}
\phantomsection\label{stargazer-multiple-models}
\tiny
\begin{table}[!htbp] \centering 
  \caption{Rendered \texttt{stargazer} Table with 2 Models} 
  \label{} 
\begin{tabular}{@{\extracolsep{5pt}}lcc} 
\\[-1.8ex]\hline 
\hline \\[-1.8ex] 
\\[-1.8ex] & \multicolumn{2}{c}{Sepal.Length} \\ 
\\[-1.8ex] & (1) & (2)\\ 
\hline \\[-1.8ex] 
 Sepal.Width & 0.651$^{***}$ &  \\ 
  & (0.067) &  \\ 
  & & \\ 
 Petal.Length & 0.709$^{***}$ &  \\ 
  & (0.057) &  \\ 
  & & \\ 
 Petal.Width & $-$0.556$^{***}$ & $-$0.311$^{***}$ \\ 
  & (0.128) & (0.114) \\ 
  & & \\ 
 Sepal.Width:Petal.Length &  & 0.185$^{***}$ \\ 
  &  & (0.017) \\ 
  & & \\ 
 Constant & 1.856$^{***}$ & 4.150$^{***}$ \\ 
  & (0.251) & (0.078) \\ 
  & & \\ 
\hline \\[-1.8ex] 
Observations & 150 & 150 \\ 
R$^{2}$ & 0.859 & 0.821 \\ 
\hline 
\hline \\[-1.8ex] 
\textit{Note:}  & \multicolumn{2}{r}{$^{*}$p$<$0.1; $^{**}$p$<$0.05; $^{***}$p$<$0.01} \\ 
\end{tabular} 
\end{table}
\end{frame}

\begin{frame}[fragile]{Stargazer -- Summary Data}
\phantomsection\label{stargazer-summary-data}
\footnotesize

\begin{Shaded}
\begin{Highlighting}[]
\FunctionTok{stargazer}\NormalTok{(iris, }\AttributeTok{type =} \StringTok{"text"}\NormalTok{, }\AttributeTok{summary =} \ConstantTok{TRUE}\NormalTok{, }\AttributeTok{title =} \StringTok{"Summary of Iris Dataset"}\NormalTok{)}
\end{Highlighting}
\end{Shaded}

\begin{verbatim}

Summary of Iris Dataset
===========================================
Statistic     N  Mean  St. Dev.  Min   Max 
-------------------------------------------
Sepal.Length 150 5.843  0.828   4.300 7.900
Sepal.Width  150 3.057  0.436   2.000 4.400
Petal.Length 150 3.758  1.765   1.000 6.900
Petal.Width  150 1.199  0.762   0.100 2.500
-------------------------------------------
\end{verbatim}
\end{frame}

\begin{frame}{Visualization Example}
\phantomsection\label{visualization-example}
\begin{itemize}
\tightlist
\item
  Using the \texttt{mtcars} dataset -- \texttt{library(mtcars)} --
  complete the following:

  \par \vspace{2.5mm}

  \begin{enumerate}
  \tightlist
  \item
    Using \texttt{mpg} as the dependent variable, compile two models
    using \texttt{cyl}, \texttt{disp}, \texttt{hp}, and \texttt{wt} --
    the second should have an interaction between \texttt{disp} and
    \texttt{wt}.

    \par \vspace{1.5mm}
  \item
    Produce a table using \texttt{stargazer} of the resulting models.

    \par \vspace{1.5mm}
  \item
    Use \texttt{ggplot} to illustrate the distribution of each of the
    variables listed in (1).
  \end{enumerate}
\end{itemize}
\end{frame}

\section{Looking Forward}\label{looking-forward}

\begin{frame}{Looking Forward}
\phantomsection\label{looking-forward-1}
\begin{itemize}
\tightlist
\item
  Homework: Problem Set (Class 2)

  \par \vspace{2.5mm}
\item
  Next Class: Parallel Computing
\end{itemize}
\end{frame}

\end{document}
